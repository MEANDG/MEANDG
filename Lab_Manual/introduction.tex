\chapter{Introduction}
\section{What}

MEAN-DG ({\bf M}ixed {\bf E}lements {\bf A}pplications with {\bf N}odal {\bf D}iscontinuous {\bf G}alerkin method) 
aims to be a versatile, industry-applications oriented efficient Discontinuous Galerkin code. 
However, all efforts have been taken to  keep it as a flexible framework such that in addition to the Galerkin family
of schemes, other schemes such as the FV scheme, spectral methods etc. can be written as required. 
The code in its finished form is expected to cater four types of 3D elements: hexahedral, tetrahedral, prism and pyramid cells.
It can be customized for a wide array of equations ranging from as simple as the linear advection equation, Laplace equation
to as complicated as the Navier Stokes (N-S) equations and Magnetohydrodynamics (MHD) equations. 
The code is mostly a c++ code with some utilities written in Python language. 
The data-structure and the file reading format is inspired from the OpenFOAM file format. Thus, test-cases
and applications for OpenFOAM can be readily solved using MEAN-DG (depending on the solvers coded).
After the current phase the code will have following features:
\begin{itemize}
	\item  Solve advection dominated flows efficiently
	\item  Achieve arbitrary higher-order accuracy in space
	\item  Open-MP parallelization 
	\item  Capability to handle flexible geometries and different mesh element types
	\item  Capability to code other systems of equations
\end{itemize}

\section{Why}
Higher-order accuracy in space is extremely important for many applications such as boundary layer flows, vortex dominated
flows, linear propagating waves (such as in electromagnetics and aeroacoustics) etc. Finite volume (FV) methods are well established
and mature enough to cater to the industry demands. However, FV methods suffer from certain limitations, such as:
\begin{itemize}
\item Poor hardware scalability and parallelization capacity (because of wider stencils used and the inter-cell communications it demands) 
\item Limitations on order of accuracy because of non-compact nature of the reconstruction
\item Difficult boundary treatments
\end{itemize}
On the other hand the spectral methods and global collocation methods are extremely accurate but lack flexibility and 
simplicity of the finite volume methods. The DG methods fall right in the sweet spot between the element based Galerkin methods
and the finite volume methods. As of now, not many large-scale codes are available based on the DG framework. A few codes are 
available such as,
\begin{enumerate}
\item Deal.II (http://www.dealii.org/) 
\item Dune (https://www.dune-project.org/about/dune/) 
\item Fenix (https://fenicsproject.org/documentation/) 
\end{enumerate}
The idea behind MEAN-DG is to have a DG code similar to the OpenFOAM FV code for industry level applications. 
It also serves the purpose of capacity building, indigenous technology building and have an in-house computation-platform.

\section{Who}
The PI of the project is Prof. S. Gopalakrishnan. Primary contributions to the project are made by Tapan Mankodi and Subodh Joshi.
The project is funded by Siemens Technologies, Bangalore. 

\section{How {\small {to refer this manual}}}
Chapter \ref{math} covers the necessary mathematical background to get started with the code. Chapter \ref{structure} explains
the structure of the code. Chapter \ref{application} explains the work-flow for a few applications solved with MEAN-DG. 
Chapter \ref{license} consists of the license. User is strongly recommended to go through it before using the code.

